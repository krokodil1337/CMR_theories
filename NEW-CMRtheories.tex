\documentclass[a4paper,reqno]{amsart}
\usepackage[english]{babel}
%\usepackage[applemac]{inputenc}
\usepackage[T1]{fontenc}
\usepackage{amsmath,amssymb,amsfonts,amsthm}
\usepackage{hyperref}
\usepackage{slashed}
\usepackage{graphicx,color}
\usepackage[bbgreekl]{mathbbol}
\usepackage{mathtools}
\usepackage[all,cmtip]{xy}


\newtheorem{definition}{Definition}
\newtheorem{theorem}[definition]{Theorem}
\newtheorem{proposition}[definition]{Proposition}
\newtheorem{lemma}[definition]{Lemma}
\newtheorem{corollary}{Corollary}



%\newenvironment{definition}{\begin{definizione}}{\newline \phantom{.}\hfill $\bullet$ \end{definizione} }

\newtheorem{remark}[definition]{Remark}

%\newenvironment{remark}{\begin{Remark}}{\newline \phantom{.} \hfill $\maltese$ \end{Remark}}
\newcommand{\g}{\mathfrak{g}}
\newcommand{\calA}{\mathcal{A}}
\newcommand{\calP}{\mathcal{P}}
\newcommand{\de}{\partial}
\newcommand{\X}[1]{(X_{#1})}
\newcommand{\HX}[1]{(\widehat{X}_{#1})}
\renewcommand{\qedsymbol}{$\checkmark$}
\newcommand{\ev}[1]{\textbf{#1}\index{#1}}
\newcommand{\qsp}[2]{\,\ensuremath{\raise.5ex\hbox{$#1$}\big\slash\raise-.5ex\hbox{$#2$}}} 
\newcommand{\smoo}{\mathcal{C}^\infty}
\newcommand{\pard}[2]{\frac{\delta#1}{\delta#2}}
\newcommand{\ber}{\mathrm{Ber}}
\newcommand{\str}[1]{\mathrm{Str}\left[#1\right]}
\newcommand{\tr}[1]{\mathrm{Tr}\left[#1\right]}
\newcommand{\trasp}[1]{\prescript{t\!}{}{#1}}
\newcommand{\rpar}[2]{\frac{#1\!\!\stackrel{\leftarrow}{\partial}}{\partial z^{#2}}}
\newcommand{\txi}[1]{\widetilde{\xi}^{#1}}
\newcommand{\tJ}[1]{\widetilde{J}_{#1}}
\newcommand{\tgd}[1]{\widetilde{g}^\dag{}^{#1}}
\newcommand{\tg}{\widetilde{g}}
\newcommand{\tgam}{\widetilde{\gamma}}
\newcommand{\sfJ}{\mathsf{J}}
\newcommand{\sfg}{\mathsf{g}}
\newcommand{\bb}{\mathsf{b}}
\newcommand{\nn}{{\underline{n}}}
\newcommand{\tgr}{\widetilde{\mathrm{g}}^\partial}
\newcommand{\tR}{\widetilde{R}}
\newcommand{\RR}{\mathbb{R}}
\newcommand{\intl}{\int\limits}
\newcommand{\tc}{\widetilde{c}}
\newcommand{\tom}{\widetilde{\omega}}
\newcommand{\bom}{\boldsymbol{\omega}}
\newcommand{\te}{\widetilde{e}}
\newcommand{\bem}{\mathbf{e}}
\newcommand{\be}{\mathbf{E}_\gamma}
\newcommand{\tedl}{\widetilde{\underline{e}}^\dag}
\newcommand{\bedl}{\mathbf{e}^\dag}
\newcommand{\bg}{\boldsymbol{\gamma}}
\newcommand{\bphi}{\boldsymbol{\varphi}}
\newcommand{\bxi}{\boldsymbol{\xi}}
\newcommand{\bpi}{\boldsymbol{\Pi}}
\newcommand{\bc}{\mathbf{c}}
\newcommand{\bt}{\mathbf{t}}
\newcommand{\bod}{\boldsymbol{\omega}^\dag}
\newcommand{\Wedge}[1]{{\textstyle \bigwedge^{#1}}}
\newcommand{\cA}{\mathcal{A}}
\newcommand{\dr}{\mathrm{d}}
\newcommand{\LQ}{\mathcal{L}_{Q}}


\title{Emergence of alternate theories in the BV-BFV setting}
\author{Pavel Mnev}


\author{Michele Schiavina}


\author{Konstantin Wernli}


\begin{document}
\maketitle
TESTESTEST
\section{Introduction}
In this paper we show how the BV-BFV approach to Field Theory on manifolds with boundaries, corners, and higher codimension strata unveils ambiguities that are intrinsic of certain field theories. Given an \emph{$n$-extended theory}, i.e. a field theory for which the BV-BFV axioms hold - recursively - up to codimension $n$, we talk about the emergence of \emph{alternate theories} if, at a given codimension $k$, there exist two naturally induced functionals both satisfying the BV-BFV axioms. When this happens, the \emph{difference} $\Delta^{(k)}$ of the two functionals becomes of central relevance.

Looking at the important example of Chern--Simons theory, we show that it admits an alternate at all codimensions $0< k < 3$, and we interpret the difference in codimension-one $\Delta^{(1)}$ in terms of \emph{gauged} Wess--Zumino--Witten theory, for a Lie group valued field $g\colon M \longrightarrow G$. On the one hand, we show that if we consider a on-parameter family $g_t=\mathrm{exp}(tA)$ for some Lie algebra valued function $A\colon M\longrightarrow \mathfrak{g}=\mathrm{Lie}(G)$ the difference $\Delta^{(1)}$ coincides with the linearised gauged Wess--Zumino action (that is its derivative with respect to the parameter $t$). Moreover, implementing a construction that stems from the AKSZ approach to field theory, we are able to promote a Lie-algebra-valued field to a Lie-group-valued one, thus reversing the (target) derivation procedure at the level of functionals over spaces of fields. If we denote by $S_{WZ}[A,g]$ the gauged Wess--Zumino action and $I\subset \mathbb{R}$ a closed interval, we have
\begin{equation}
S_{WZ}[A,g]=T_{I}^0 \Delta^{(1)}\vert_{\mathrm{dgMap}^0}
\end{equation}
where $T_{I}^0$ is the transgression map (on $0$-forms) coming from the AKSZ procedure,
\begin{equation}
\xymatrix{
\mathrm{Map}(T[1]I, \mathcal{F}^{(1)})\times T[1]I \ar[d]_{p}   \ar[r]_-{\mathrm{ev}}  & \mathcal{F}^{(1)}\\
\mathrm{Map}(T[1]I, \mathcal{F}^{(1)}) & 
}
\end{equation}
with $\mathcal{F}^{(1)}$ the space of Chern--Simons codimension-$1$ fields, the map 
$$T^\bullet_I\colon\Omega^\bullet\left(\mathcal{F}^{(1)}\right)\longrightarrow \Omega^\bullet\left(\mathrm{Map}(T[1]I, \mathcal{F}^{(1)})\right)$$ 
given by the composition $T^\bullet_I:=p_* \mathrm{ev}^*$, and $\mathrm{dgMap}^{0}$ denoting the set of degree-$0$, differential graded maps in $\mathrm{Map}(T[1]I, \mathcal{F}^{(1)})$.

To recover the \emph{Witten} part of the celebrated (chiral) Wess--Zumino--Witten functional, we choose a (metric dependent) polarisation $\mathcal{P}_\eta$ in $\mathcal{F}^{(1)}$ and, by changing the data consistently, we are able to show that
\begin{equation}
S_{WZW}[A,g;\eta]=T_{I}^0 \Delta^{(1)}_{\mathcal{P}_\eta}\vert_{\mathrm{dgMap}^0}
\end{equation}
with $\eta$ a choice of a metric on $M$. {\color{red} Maybe a conformal structure is enough?}

\section{Main Construction}
We will be concerned with a Lagrangian version of the classical BV-BFV approach to field theories, as presented in \cite{CMR1}. The \emph{space of fields} should be thought of as the space of $C^\infty$ sections of a (graded) fiber bundle over an $m$-dimensional spacetime manifold $M$. We allow boundaries and higher codimension data like corners, corners with boundaries, \emph{et cetera}. We denote by $\{M^{(k)}\}_{k=0\dots m}$ a stratified manifold, i.e. sumbanifolds with inclusions $\iota_k\colon M^{(k)} \longrightarrow M^{(k-1)}$, however for the most part we will not be concerned with such data directly. As a matter of fact, we will consider local functionals and local forms with values in inhomogeneous differential forms on $M$ (see Definition \ref{localforms} below), so that specifying higher codimension submanifold data in $M$ will allow us to integrate the aforementioned forms and obtain the usual BV-BFV data.

%\item $\sim^{(k)}$ is an equivalence relation of codimension-$k$ submanifolds in $M$, i.e. $\iota_k \colon M^{(k)} \longrightarrow M^{(k-1)}$,

\begin{definition}\label{localforms}
A \emph{local} form on $E\longrightarrow M$, a fiber bundle on an $m$-dimensional $M$, is an element of 
\begin{equation}
\Omega^{\bullet,\bullet}_{\mathrm{loc}}(\mathcal{E}\times M,\delta,d) := (j^\infty)^*\Omega^{\bullet,\bullet}(J^\infty (E),d_V,d_H)
\end{equation}
with $\mathcal{E}:=\Gamma^\infty(M,E)$ and $j^\infty$ the limit of the maps $\{j^p\colon \mathcal{E}\times M \longrightarrow J^pE\}$. An element of $\Omega_{\mathrm{loc}}^{0,\bullet}(\mathcal{E}\times M)$ is called \emph{Local Functional}.
\end{definition}

\begin{definition}
An n-extended, exact BV-BFV manifold, shorthanded with \emph{$n$-extended theory}, over the $m$-dimensional stratified manifold $\{M^{(k)}\}_{k=0\dots m}$ ($m\geq n$) is the data 
$$\mathfrak{F}^{\wedge n}=(\mathcal{F}^{(k)}, \alpha^{(k)},L^{(k)}, Q)_{k=0\dots n},$$
where, for every $k\leq n$
\begin{itemize}
\item $\mathcal{F}\equiv\mathcal{F}^{(0)}$, the \emph{space of fields}, is the space of sections of the bundle (or sheaf) $E\longrightarrow M^{(0)}$, and $F^{(k)}$ is the space of restrictions of fields in $\mathcal{F}$ to a tubular neighborhood of the k-stratum $M^{(k)}$,
\item $\omega^{(k)}\coloneqq\delta\alpha^{(k)}\in\Omega_{\mathrm{loc}}^{2,k}(\mathcal{F}\times M)$ is pre-symplectic on $F^{(k)}$ , i.e. $\emph{ker}(\omega^{(k)}\vert_{F^{(k)}}^\sharp)$ is a subbundle of $TF^{(k)}$,
\item $Q$ is a degree-1, local, evolutionary, cohomological vector field on $\mathcal{F}$, i.e. $[Q,Q]=[Q,d]=0$, 
\item $L^{(k)}\in \Omega_{\mathrm{loc}}^{0,m-k}(\mathcal{F}\times M)$ is a degree-$k$ local functional,
\end{itemize}
such that
\begin{subequations}\label{CMReqts}\begin{align}
&  \iota_{Q} \Omega^{\bullet} = \delta L^{\bullet} + d\alpha^{(\bullet + 1)}\\\label{bracket}
&\frac12 \iota_Q\iota_Q\Omega^\bullet = d L^{(\bullet+1)}.
\end{align}\end{subequations}
When $n=\mathrm{dim}(M)$ we say that the theory is \emph{fully extended}.
\end{definition}

%; for $n=0$ the data defines a BV-manifold, and for $n=1$ the data is that of an exact BV-BFV manifold.

\begin{definition}[{\color{blue} Check this, it is of course a sketch. It allows us not to worry about what "class" of bordisms we are considering right away.}]
A BV-BFV theory is the (functorial?) assignment of a BV-BFV manifold over $\{M^{(k)}\}$ for all stratified manifolds {\color{blue} in a suitable equivalence class (COMMENT: I am actually not sure that we need to separate the pseudo-Riemannian case in how we treat strata. The condition of "space-like" boundaries should be thought of at the level of fields, not the underlying topological spaces. This is how we dealt with Einstein Hilbert: an open condition in the space of Pseudoriemannian metrics on M)}
\end{definition}


\begin{remark}\label{otherrelations}
Notice that from equations \eqref{CMReqts} the following relations follows (recall that $[Q,d]=0$ and $[\delta,d]=0$): 
\begin{subequations}\begin{align}\label{mCME}
\LQ\left(L^{\bullet}\right)& = dL_{CMR}^{(\bullet+1)}\\ \label{omegacocycle}
\LQ \Omega^\bullet & = d\Omega^\bullet 
\end{align}\end{subequations}
where we adopted the notation\footnote{We call $L_{CMR}$ the CMR-Lagrangian, where CMR stands for Cattaneo, Mn\"ev, Reshetikhin.} $L^{(k+1)}_{CMR}\coloneqq \left(2L^{(k+1)} - \iota_{Q^{(k+1)}}\alpha^{(k+1)}\right)$. Because $Q$ also encodes the symmetry data of the theory, Equation \eqref{mCME} roughly measures the failure of gauge invariance of the Lagrangian in the presence of higher codimension data. We will make this statement more precise in what follows. 
Observe that Eq. \eqref{omegacocycle} means that $\Omega^\bullet$ is an $(\LQ - d)$-cocycle, implying that 
$$\delta (\LQ - d)\alpha^\bullet=0$$ 
is $\delta$-closed. 

\end{remark}

%%%I redefined L^k -> (-1)^{m-k}L^k and \omega -> (-1)^{d-k}\omega 

\begin{remark}
In practical situations Q is defined (locally) on \emph{fields}, and extended by prolongation to the jet bundle to obtain a legitimate evolutionary vector field.
\end{remark}



\begin{definition}
For every $k\leq m$, the complex $\Omega^{\bullet,\bullet}_{\mathrm{loc}}(\mathcal{F}\times M)$ admits a third differential given by the Lie derivative $\LQ$. We define the BV-BFV complex ({\color{red} This name is temporary?!}) to be the space of local forms endowed with the combined differential $\LQ - d$:
\begin{equation}
\Omega_{\mathrm{BV-BFV}}^{\bullet}(\mathcal{E}\times M, \LQ - d)
\end{equation}
We will use the shorthand notation $\mathbb{\Omega}^{\bullet}(\LQ - d)\equiv\Omega_{\mathrm{BV-BFV}}^{\bullet}(\mathcal{F}\times M)$, and the cohomology is denoted by $\mathbb{H}^\bullet(\LQ - d)$.
\end{definition}





\begin{remark}\label{polarisationremark}
It is possible to modify the codimension-$k$ Lagrangian by a $d$-exact term $L^{(k)} \mapsto L^{(k)} + df^{(k+1)}$, where $f^{(k+1)}$ is a degree-$k$ local functional, and compensate this with a shift\footnote{Observe that both $\alpha^{(k+1)}$ and $f^{(k+1)}$ have the same degree, $k$.} of $\alpha^{(k+1)}$ by $\delta f^{(k+1)}$, to preserve equations \eqref{CMReqts}. This is related to the introduction of a polarisation $\mathcal{P}$ on $\mathcal{F}^{(k+1)}$ - the space of fields of a codimension $k+1$-stratum of $M$, in order to have $\alpha^{(k+1)}$ vanish on the fibers of the polarisation, as required by the quantisation procedure developed by CMR \cite{CMR2}. In that case we denote the adapted (polarised) one-form and Lagrangian functional by $\alpha^{(k+1)}_{\mathcal{P}},L^{(k)}_{\mathcal{P}}$. The new, polarised, CMR Lagrangian is clearly 
$$L^{(k+1)}_{CMR:\mathcal{P}}=L^{(k+1)}_{CMR} - Q (f^{(k+1)}).$$
\end{remark}

\subsection{Alternates}

It is reasonable to ask whether $L^{\bullet}$ and $L^{\bullet}_{CMR}$ differ, and how. There are examples in which that is not the case for at least one stratum, like in abelian BF theory, and that is when $\iota_{Q}\alpha^{\bullet} = L^{\bullet}$. We can then consider the differences
\begin{equation}
\Delta^{\bullet} := L^{\bullet}_{CMR} - L^{\bullet};\ \ \Delta^{\bullet}_{\mathcal{P}} := L^{\bullet}_{CMR:\mathcal{P}} - L^{\bullet}.
\end{equation}


\begin{lemma}\label{Deltacocycle}
The difference $\Delta^\bullet$ is an $(\LQ - d)$ cocycle, and its $\delta$-variation vanishes.
\end{lemma}

\begin{proof}
This is a consequence of Eq. \eqref{CMReqts}, since
\begin{multline}
\delta L^\bullet = \iota_Q\delta\alpha^{\bullet} - d\alpha^{(\bullet+1)} = \LQ\alpha^\bullet + \delta\iota_Q\alpha^\bullet - d\alpha^{(\bullet+1)} \iff \\
\delta (L^\bullet-\iota_Q\alpha^\bullet) = (\LQ - d)\alpha^\bullet \iff (\LQ - d)\delta \Delta^\bullet = (\LQ - d)^2\alpha^\bullet\iff \delta (\LQ - d)\Delta = 0
\end{multline}
However, $\Delta^{(k)}$ is a degree $k$-local functional ($k\geq 0$) and $\LQ \alpha^{(k)} - d\alpha^{(k+1)}$ is then of degree $k+1 \geq 1$, implying that, as there are no nonzero degree constants, $(\LQ -d)\alpha^\bullet=0$.
\end{proof}


{\color{blue}The following definitions need to be checked and rearranged, because of Lemma \ref{Deltacocycle}.}





\begin{definition}[{\color{red} TO BE CHECKED, maybe we need a definition for when $\Delta$ is a coboundary, if Lemma \ref{Deltacocycle} is true. Also the term closable is nice, because it is about making $L^\bullet$ closed under $\LQ -d$.} ]
An n-extended theory is said to be \emph{closable} iff the differences at $k$ are cocycles $\Delta^{(k)}\in \mathbb{H}^{k}\left(\LQ-d\right)$ for all $k\leq n$.
\end{definition}


\begin{definition}[{\color{red} This is sort of a negative definition with respect of my comment above. We might want to reformulate this in a different way, I don't know.}]
An n-extended theory $\mathfrak{F}^{\wedge n}$ is called \emph{unambiguous at codimension $k$} if, for $k\leq n$ there exists a degree-$(k-1)$ functional $f^{(k-1)}$ on $\mathcal{F}$ such that $\Delta^{(k)} = (\LQ-d)f^{(k-1)}$. It is said to be fully unambiguous if the condition holds for all $k\leq n$.
\end{definition}

\begin{definition}
An n-extended theory $\mathfrak{F}^{\wedge n}$ is said to admit an \emph {alternate at condimension $k\leq n$} if $L^{(k)}_{CMR}$ satisfies the $k$-modified classical master equation. It is called \emph{fully alternate} if the condition holds for all $0<k\leq n$.
\end{definition}


\begin{remark}
Notice that the condition $\Delta^{(k)}=(\LQ-d)f^{(k-1)}$ is relevant because if the theory is unambiguous, but admits an alternate at codimension $k$, we have that $L^{(k)}_{CMR}=L^{(k)}$ and $\Delta^{(k)}$ is interpreted as a \emph{polarisation term}, explaining why $L^{(k)}_{CMR}$ satisfies the mCME. On the contrary, when $\Delta^{(k)}$ is not exact we cannot interpret it as a polarisation term (cf. Remark \ref{polarisationremark}), and if it defines an alternate something new is happening.
\end{remark}


\begin{theorem}
Let $\mathfrak{F}^{\wedge n}$ be a {\color{red}closable (not needed anymore?)}, n-extended theory.  
\begin{equation}
\mathbb{L}\coloneqq \sum_{k=0}^n L^{(k)} + k \Delta^{(k)}
\end{equation}
is a $\left(\LQ-d\right)$-cocycle.
\end{theorem}

\begin{proof} Since $\mathfrak{F}^{\wedge n}$ is n-extended we know that, for all $k\leq n$, $\LQ L^{(k)}=d L_{CMR}^{(k+1)}$. Moreover, $\Delta^{(k)}$ is a $(\LQ-d)$-cocycle for all $k$ ({\color{blue}because it is closable - probably not needed anymore}). We use this and $\Delta^{(k)}=L_{CMR}^{(k)} - L^{(k)}$ to compute:
\begin{multline*}
\LQ\left(\sum_{k=0}^n L^{(k)} + k \Delta^{(k)}\right) = \sum_{k=0}^n \left( \LQ L^{(k)} + k\LQ\Delta^{(k)}\right)= \sum_{k=0}^n\left(dL_{CMR}^{(k+1)} + k d\Delta^{(k+1)}\right) \\
 = \sum_{k=0}^n\left(d\left(L^{(k+1)}+\Delta^{(k+1)}\right) + k d\Delta^{(k+1)}\right) = \sum_{k=0}^n d \left( L^{(k+1)} + (k+1) \Delta^{k+1}\right)
\end{multline*}
\end{proof}


\subsection{Chern--Simons Theory}
Chern--Simons theory is a fully extended field theory for Lie-algebra-valued inhomogeneous forms $\cA\in\Omega^\bullet(M)[1 - \bullet]\otimes\mathfrak{g}$ (called \emph{superfield}) on a three-dimensional manifold $M$. The zero-degree part of the theory is the usual Chern--Simons theory of connections on a (trivial) principal bundle $P\longrightarrow M$. The following definition fixes the notation:

\begin{definition}
Fully extended Chern--Simons theory for the {\color{blue}compact? unimodular? semisimple?} Lie group $G$, with $(\mathfrak{g},\ [\cdot,\cdot],\ \langle \cdot,\cdot \rangle)$ its Lie algebra endowed with an invariant inner product, is the data
$$\mathfrak{F}_{CS}=\left(\Omega^\bullet(M)[1-\bullet]\otimes\mathfrak{g}, \alpha_{CS}^{\bullet}, L^{\bullet}_{CS}, Q_{CS}\right)$$
where the Lagrangian functional and the one-form $\alpha^{(k)}$ in codimension $k\leq \mathrm{dim}(M) = 3$ for the superfield $\cA\in\Omega^\bullet(M)[1 - \bullet]$ read, respectively
\begin{subequations}\begin{align}\label{CSaction}
L^{(k)}_{CS}[\cA]  & = \left[\frac12 \langle\cA, d\cA\rangle + \frac16 \langle\cA, [\cA,\cA]\rangle\right]^{(k)}\\
\alpha^{(k)}[\cA] & =  \mathrm{Tr} \left[\langle\cA, \delta\cA\rangle\right]^{(k)}
\end{align}\end{subequations}
where the apex ${}^{(k)}$ means that we are taking the $k$-form part of the inhomogeneous local forms, and the vector field $Q$ acts on fields as 
\begin{equation}
Q\cA = F_{\cA} = d\cA + \frac12 [\cA,\cA] 
\end{equation}
We shall understand the trace symbol from now on. 
\end{definition}


\begin{proposition}[{\color{blue}We might want to fix the wording here. The last part of the statement is lacking a proof.}]
The fully extended theory $\mathfrak{F}_{CS}$ admits a $(1)$-alternate. Moreover, the BV-BFV extension of the alternate theory $L^{(1)}_{CMR}$ is strongly equivalent to the CMR action in codimension 2.
\end{proposition}

\begin{proof}
In codimension $k=1$ we have
\begin{equation}
L^{(1)}_{CS}= cF_A + \frac12 A^\dag [c,c] - \frac12 d(Ac)
\end{equation}
and 
\begin{equation}
\Omega^{(1)} = (-1)^{3-1}\delta \alpha^{(1)} = \frac12\delta \left( A\delta A + A^\dag \delta c + c\delta A^\dag\right) =  \frac12 \delta A\delta A + \delta A^\dag\delta c
\end{equation}

The codimension-$1$ CMR action reads
\begin{equation}
L^{(1)}_{CMR}=cF_A + \frac14 A^\dag[c,c] - \frac14[A,A]c - \frac12 d(Ac)
\end{equation}

We choose a polarisation in the space of codimension 1 fields and we modify the boundary one form so that it vanishes on the (Lagrangian) fibers of this polarisation. We choose here a splitting of the connection field $A$ into its holomorphic and anti-holomorphic parts (resp. $A^{1,0}$ and $A^{0,1}$), and the fibers of the polarisation will be defined by $A^{1,0}$ and $c$ constant. This requires the modification of the functional $L^{(1)}$ by the functional $f_{1,0}\coloneqq \frac12\left( A^{1,0}A^{0,1} + cA^\dag \right)$ and produces the shift 
$$\alpha^{(1)}_{1,0}\coloneqq \alpha^{(1)} + \delta f_{1,0}= A^{0,1}\delta A^{1,0} + A^\dag \delta c,$$
which clearly vanishes on fibers with constant $A^{1,0}$ and $c$. Moreover, with
\begin{align*}
Qf_{1,0} =&  \frac12\left[ \partial_{A^{1,0}}c A^{0,1} - A^{1,0}\overline{\partial}_{A^{0,1}}c - \frac12[c,c] A^\dag - cF_A\right] \\
=& - A^{1,0}\overline{\partial}_{A^{0,1}}c - \frac14[c,c]A^\dag - \frac12 c[A^{1,0},A^{0,1}]
\end{align*}
we can then compute the codimension-$1$ polarised CMR action, which reads
\begin{equation}
L^{(1)}_{CMR:1,0}=A^{1,0} \overline{\partial} c + cF_A + \frac12A^\dag[c,c] - \frac12d(Ac)
\end{equation}

The difference $\Delta^{(1)}$ and its polarised version $\Delta^{(1)}_{1,0}$ are
\begin{equation}
\Delta^{(1)}= -\frac14 \left( [A,A]c + [c,c]A^\dag\right);\ \ \ \ \Delta^{(1)}_{1,0}=A^{1,0} \bar{\partial} c 
\end{equation}
and if we compute the cohomological vector field associated with the (polarised) CMR action we obtain (dropping the sup- and subscripts for convenience)
\begin{equation}
Q^{(1)}_{CMR:1,0} :\begin{cases}
Q(c) = \frac12 [c,c]\\
Q(A^{1,0}) = \partial_{A^{1,0}} c\\
Q(A^{0,1})= \bar{\partial}c + \bar{\partial}_{A^{0,1}} c\\
Q(A^\dag)= F_A + \bar{\partial}A^{1,0} + [c,A^\dag]
\end{cases}
\end{equation}
It is then straightforward to check that $[Q^{(1)}_{CMR:1,0},Q^{(1)}_{CMR:1,0}]=0$, and that Chern--Simons theory admits a $1$-alternate, because $\Delta^{(1)}$ is a nontrivial cocycle. In fact we can write
\begin{equation}
\Delta^{(1)}=- \frac{1}{12} \left[\mathcal{A}[\mathcal A, \mathcal{A}]\right]^{(1)}
\end{equation}
then we have the short computation
\begin{multline}
\langle F_{\mathcal{A}}, [\mathcal{A},\mathcal{A}] \rangle= \langle d_{\mathcal{A}}\mathcal{A}, [\mathcal{A},\mathcal{A}] \rangle - \langle[\mathcal{A},\mathcal{A}], [\mathcal{A},\mathcal{A}] \rangle=\langle d\mathcal{A}, [\mathcal{A},\mathcal{A}]\rangle = \frac13 d \langle\mathcal{A}, [\mathcal{A},\mathcal{A}]\rangle
\end{multline}
so that 
$$Q\Delta^{(1)} = -\frac14 \langle F_{\mathcal{A}},[\mathcal{A},\mathcal{A}]\rangle =-\frac{1}{12} d \langle \mathcal{A},[\mathcal{A},\mathcal{A}]\rangle$$
which again shows that $\Delta$ is a $(\LQ - d)$-cocycle. To show that it is nontrivial we pick a polynomial $F= \sum_{p}f_{2p+1}\mathcal{A}^{2p+1}$ and apply $\LQ-d$ to it:
\begin{multline}
(\LQ-d)F = \sum_p (2p+1) f_{2p+1}(F_{\mathcal{A}} - d\mathcal{A})\mathcal{A}^{2p} = \sum_p (2p+1) f_{2p+1}\frac12[\mathcal{A},\mathcal{A}]\mathcal{A}^{2p}
\end{multline}
which can only be an even polynomial in $\mathcal{A}$ and therefore will never equal $\Delta^{(1)}$.

\end{proof}







\subsection{Poisson $\Sigma$-model}

\section{Obtaining WZW from Chern-Simons theory}
 \begin{lemma}
 Let $S$ denote the BV-Chern-Simons action and let $g \in C^{\infty}(M,G)$ be a large gauge transformation. Then, we have $$S[\calA^g] - S[\calA] = S[A^g] - S[A] = \frac{1}{2}\int_{\de M} \langle g^{-1}Ag, g^{-1}dg \rangle -\int_M \frac{1}{12}\langle g^{-1}dg,[g^{-1}dg,g^{-1}dg]\rangle$$. 
\end{lemma}
\begin{proof}
Proof is just computations, how much to put?
\end{proof}
\subsection{Polarisations} 
Now assume that we have chosen a polarisation $\calP$ on the space of boundary fields (Lagrangian, smooth leaf space) such that there exists $f^{\calP}$ such that $\alpha^{\de,\calP} := \alpha^{\de} + \delta f^{\calP}$ restricts to 0 on fibers of $\calP$\footnote{it would be good to understand a bit better why we want that.} In that case we change the action to $S^{\calP} = S + \pi^*f$. \emph{The boundary action does not change!} The CMR action changes to 
$$S_{CMR}^{\calP} = S_{CMR} -Q^{\de}f^{\calP}.$$ 
\subsubsection{Complex Polarisation}
One way\footnote{Are there other known ways?} of choosing a polarisation is to choose a complex structure on the boundary surface $\Sigma $. That way one can split the space of 1-forms on $\de M$ into its Dolbeaut parts $\Omega^1(\Sigma) = \Omega^{(1,0}(\Sigma) \oplus \Omega^{(0,1)}$, where $\Omega^{(1,0)}(\Sigma)$ is the space of 1-forms that locally look like  $ f(z,\bar{z})dz $. Then the space of boundary fields splits as 
$$ \Omega^{\bullet}(\Sigma,\g) = \Omega^0(\Sigma,\g) \oplus \Omega^{(1,0)}(\Sigma,\g) \oplus \Omega^{(0,1)}(\Sigma,\g) \oplus \Omega^{2}(\Sigma,\g) \ni c + A^z + A^{\bar{z}} + A^+.$$
We will choose the $(c,A^z)$ polarisation and choose $f$ as 
\begin{align*}
f^{\calP} &= \frac{1}{2} \int_M  \langle A^z, A^{\bar{z}} \rangle + \langle c, A^+ \rangle 
\end{align*}
The adapted 1-form is 
\begin{align*} 
\alpha^{\de,\calP} &= \alpha^{\de} + \delta f  = \frac{1}{2} \int_{\de M}\langle A^z,\delta A^{\bar{z}} \rangle + \langle A^{\bar{z}} , \delta A^z \rangle + \langle c, \delta A^+ \rangle + \langle A^+, \delta c \rangle \\
&+ \frac{1}{2}\int_{\de M} \langle \delta A^z, A^{\bar{z}} \rangle - \langle A^z ,\delta A^{\bar{z}} \rangle + \langle \delta c , A^+ \rangle - \langle c, \delta A^+ \rangle \\ 
&= \int_{\de M} \langle A^{\bar{z}} ,\delta A^z \rangle + \langle A^+, \delta c \rangle \\
\end{align*}
which indeed vanishes on fibers of $\calP$ (specified by constant $c$, $A^z$). Notice also that $\delta$, like all the fields, has total degree 1.  
\subsubsection{CMR action}
Now we compute the $CMR$ action: 
\begin{align*} S_{CMR}^{\calP} &= 2S^{\de} - \iota_{Q^{\de}}\alpha^{\de} = \int_{\de M} 2 \langle c, F_A \rangle + \langle A^+ ,[c,c] \rangle  - \langle A^{\bar{z}}, d_Ac \rangle - \frac{1}{2} \langle A^+, [c,c] \rangle \\ 
&= \int_{\de M} \langle c , 2F_A - d_AA^{\bar{z}} \rangle + \frac{1}{2}\langle A^+,[c,c] \rangle  = \int_{\de M} \langle c,F_A\rangle  + \langle c, \bar\partial A^z \rangle + \frac{1}{2} \langle A^+,[c,c] \rangle \\ 
&= \int_{\de M} \langle A^z, \bar\partial c\rangle + \langle c,F_A\rangle + \frac{1}{2} \langle A^+,[c,c] \rangle 
\end{align*}
where we have integrated by parts several times and used $\dr = \partial + \bar\partial$, $[A,A^z] = 1/2 [A,A]$. 
\subsubsection{Gauge transformations} 
Under gauge transformations the $1,0$ and $0,1$ parts of $A$ transform as 
\begin{align*}
(A^z)^g &= g^{-1}A^zg + g^{-1}\partial g \\
(A^{\bar{z}})^g &= g^{-1}A^{\bar{z}}g + g^{-1}\bar\partial g \\
\end{align*}
Under the gauge transformation the adapted action behaves as 
$$S^f[\calA^g]-S^f[\calA] = S^f[A^g]-S^f[A] = \int_{\de M} \langle g^{-1}A^zg, g^{-1}\bar\partial g \rangle + \frac{1}{2} \langle g^{-1} \partial g, g^{-1}\bar\partial g\rangle - WZW.$$
\begin{proof}
We have 
\begin{align*}
 f[\calA^g] - f[\calA] &= \int_{\de M} \frac{1}{2}\langle (A^z)^g,(A^{\bar{z}})^g\rangle + \int_{\de M} \langle c^g,(A^+)^g \rangle  - \int_{\de M} \frac{1}{2}\langle A^z,A^{\bar{z}} \rangle - \int_{\de M} \langle c,(A^+) \rangle
\end{align*}
Again, the term depending on the fields of nonzero ghost number is gauge invariant, and vanishes. From the other term we get 
\begin{align*} f[\calA^g] - f[\calA] &= +\int_{\de M} \frac{1}{2}\left( \langle g^{-1}A^zg, g^{-1}A^{\bar{z}}g\rangle +  \langle g^{-1}\partial g, g^{-1}A^{\bar{z}}g\rangle + g^{-1}A^zg, g^{-1}\bar\partial g \rangle + \langle g^{-1} \partial g, g^{-1}\bar\partial g \right) \\
 &- \int_{\de M} \frac{1}{2}\langle A^z,A^{\bar{z}} \rangle \\ 
 &=   \frac{1}{2}\int_{\de M} \langle g^{-1}\partial g, g^{-1}A^{\bar{z}}g\rangle + \langle g^{-1}A^zg, g^{-1}\bar\partial g \rangle + \langle g^{-1} \partial g, g^{-1}\bar\partial g\rangle   
\end{align*} 
Now, since $S^f = S + f$ we get 
\begin{align*}
S^f[\calA^g]-S^f[\calA] &= S^f[A^g]-S^f[A] = S[A^g]- S[A] + f[A^g]-f[A] \\
&= \int_{\de M} \frac{1}{2}\langle g^{-1}Ag, g^{-1}dg \rangle - WZW  \\ 
&+ \frac{1}{2}\int_{\de M} \langle g^{-1}\partial g, g^{-1}A^{\bar{z}}g\rangle + \langle g^{-1}A^zg, g^{-1}\bar\partial g \rangle + \langle g^{-1} \partial g, g^{-1}\bar\partial g\rangle  \\ 
&= \int_{\de M} \langle g^{-1}A^zg, g^{-1}\bar\partial g \rangle + \frac{1}{2} \langle g^{-1} \partial g, g^{-1}\bar\partial g\rangle - WZW
\end{align*}
\end{proof}
Now, if we assume that $g_t$ is a time-dependent family of gauge transformations starting at $g_0 = \mathrm{id}$, we can write 
$$\frac{d}{dt} S^f[\calA^{g_t}] - S^f[\calA^{g_0}] = \int_{\de M} A^z\bar\de(\dot{g}_tg_t^{-1}).$$
{\color{red}Why Is this true when $t$ is not zero? I know we discussed it many times but I cannot remember whether we had an argument for that or not.}
\end{document}